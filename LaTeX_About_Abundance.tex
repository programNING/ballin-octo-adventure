\documentclass[11pt, letterpaper]{article}

\usepackage{graphicx, amsmath, amsfonts, amssymb, color, latexsym, multicol}
\usepackage[latin1]{inputenc}
\usepackage{mathpazo}
\usepackage[T1]{fontenc}

\usepackage[top=1in, bottom=1in, left=1in, right=1in]{geometry}

\setlength{\parskip}{.15cm}
\setlength{\parindent}{0.0cm}

\begin{document}

\section{PROTON MASS FRACTION}

The mass fraction of a proton is

\begin{equation} 
X_p = \frac{\sum_{i}Z_iY_i}{\sum_{i}A_iY_i} 
\end{equation}

where:
\begin{multicols}{2}
\begin{center}

$X_p =$ mass fraction of a proton

$Z_i =$ atomic (proton) number

\columnbreak

$A_i =$ mass number

$Y_i$ = abundance

\end{center}
\end{multicols}

\indent The numerator is the sum total of protons in every nuclide present in some matter(in our case, the NS-NS merger ejecta undergoing r-process), while the denominator is the sum total of nucleons (protons and neutrons) in every nuclide present in the matter. It is important to note that: 
\begin{equation} 
\sum_{i}A_iY_i = \sum_{i}X_i
\end{equation}
because the sum total of nucleons is essentially the total mass of all nuclides in the matter. The mass fraction of every nuclide, summed up, is 1, or 100\% of the nuclides present. Thus, 
\begin{equation} 
\sum_{i}X_i = 1
\end{equation}

\indent Matter tends to be electrically neutral, because charged particles experience electromagnetic force that quickly corrects charge imbalances. This means that there are as many electrons as there are protons. In other words, 
\begin{equation} 
Y_e = \sum_{i}Z_iY_i
\end{equation}
where $Y_e$ is abundance of electrons. Abundance describes the number of a given particle within the matter being studied. Abundance changes only when the particle is destroyed or created, and is not affected by changes in the density of the matter$^{1}$. Thus, the abundance of electrons is equal to the sum total of protons in a given matter. Recalling equations 2 and 3, electron abundance can also be written as: 
\begin{equation} 
Y_e = \frac{\sum_{i}Z_iY_i}{\sum_{i}A_iY_i}
\end{equation}
\begin{equation} 
Y_e = X_p
\end{equation}
Thus, we can calculate the electron abundance, which increases during the r-process due to beta decay of a neutron to a proton \[n \rightarrow p + e^- + \bar v_e\] from equation 1.

\section{Using Equation to Confirm Skynet's $Y_e$ Values}

\indent We will confirm that SkyNet's output values for $Y_e$ match those calculated from equation 1. In order to do that, we need values for $A_i$, $Z_i$, and $Y_i$. SkyNet provides these values for its example r-process run in an HDF5 dataset, SkyNet\_r-process\_python.h5.

\indent I copied the aforementioned file from Fermi to my personal computer, where I wrote a program to read the data, calculate $Y_e$, and plot it alongside SkyNet's values. This program, confirmYe.py, does the following things:

\indent \textit{Read data from the .h5 file.} Using h5py, a program designed that allows Python interface with HDF5 files, I opened the file and assigned the datasets of interest ($A$, $Z$, $Y_e$, time) to numpy arrays.

\indent \textit{Calculated $Y_e$ for all time steps.} Using numpy matrix operations, I used equation 1 to calculate $Y_e$. More specifically, I used numpy.dot() to perform the dot product of the abundance array (which was a 2147x7841 array) and the nucleon/proton number arrays (2147x1 arrays). This is equivalent to summing $Z_iY_i$ and $A_iY_i$ for all nuclides for all timesteps. I then divided the resulting proton  $\cdot$ abundance array by the nucleon $\cdot$ abundance array to calculate $X_p = Y_e$.

\indent\textit{Plotted the results next to SkyNet's results.} Using matplotlib pyplot commands, I was able to plot the two values of $Y_e$ on the same plot. The legend was hard to place (the loc parameter was hard to describe because of the logarithmic x-axis), and is situated awkwardly next to the y-axis.

\indent I am still facing problems with the legend placement, and saving the generated figure (the file, when saved, appears blank). Also, I don't know if I'm uploading things to GitHub correctly.

\end{document}
